\section{Conclusion}

We have shown that a system of cooperating web clients can reduce load on a origin web server by automatically switching
from client-server file transfer to peer-to-peer content delivery. Our system has shown itself to be far more scalable 
than a traditional client-server download and can significantly
reduce download times.  The system is most effective for small files but does not yet compete well with
BitTorrent for large files.  We have also found good settings for various system parameters.

Several aspects could be improved.  For example we currently connect to the origin
server from each peer, at times causing the same blocks to be served (redundantly) to various peers, slowing down the origin.
We also do not validate the integrity of downloads.  We assume peer trustworthiness and no file corruption.  We also do not provide peer incentives for sharing.

Peer lists could be optimized.  Because they are returned in chronological order from Bamboo (oldest first), our system can experience problems if peers
go offline without removing themselves from lists, or if linger times are close to expiring.  This also causes peers to download blocks 
from the the oldest 10 peers listed, which can cause unfairness. In the case of a slow Internet connection, linger times may have
already expired before peers receive the list.
Under high loads DHT response times increase substantially.  For example, if there is an extremely popular file, the 
requests for that peer list might overwhelm the DHT member responsible.  There is also some redundancy in peer lists currently.  
With large files, clients do requests for several different lists, one per block.  These lists might be all the same currently, thus
there is some avoidable redundancy and latency.

Currently this system is mostly useful for overloaded sites with static pages.  There could be some way for peers to deal with dynamic pages, such as to
store meta-data about which files are (or appear to be) static, and which are dynamic.  
Our system also only helps with bandwidth bound servers.  A way to distribute CPU load to clients would be useful.

Currently we only set static system parameters, like $R$ and $W$, etc.  This may be non optimal in actual use.