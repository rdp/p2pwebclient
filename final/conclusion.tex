\section{Conclusion}

We have shown that a system of cooperating web clients can reduce load on a origin web server by automatically switching
from client-server file transfer to peer-to-peer content delivery. Our system has shown itself to be far more scalable 
than a traditional client-server download and can significantly
reduce download times.  The system is most effective for small files but does not yet compete well with
BitTorrent for large files.  We have also found good settings for various system parameters.

Several further optimizations are unexplored.  For example we currently connect to the origin
server from each peer, at times causing the same blocks to be served (redundantly) to the many peers.

We do not validate the integrity of downloads.  We assume peer trustworthiness and no file corruption.  We also do not provide peer incentives for sharing.

Peer lists as we use them are sometimes unoptimal.  Because peer lists are always returned in chronological order from Bamboo (oldest first), our system can experience problems if peers
go offline without removing themselves from lists, or when linger times expire.  This ordering also causes peers to download blocks 
from the the oldest 10 peers listed, which can cause unfairness. It can also cause peers to receive lists which
are already outdated if they have a slow Internet connection and linger times have already expired.  A better algorithm could be designed than what we use.
Also, under high loads DHT response times increase.  For example, if there is an extremely popular file, the requests for that peer list might overwhelm the DHT member resonsible. 
This may be avoidable.

With large files, over time clients do many requests for the various blocks.  These requests might all return the same
peer list currently.  This redundancy might be avoidable.

Currently this system is mostly useful for overloaded sites with static pages.  There could be some way for peers to deal with dynamic pages, such as to
store meta-data about which files are (or appear to be) static, and which are dynamic.  
Our system also only helps with bandwidth bound servers.  A way to also distribute CPU load itself to clients would be useful.

Currently we only use static system parameters, like $R$ and $W$, etc.  This may be non optimal, though for a few values we have discovered optimal values.